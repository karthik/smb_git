\documentclass[]{article}
\usepackage[T1]{fontenc}
\usepackage{lmodern}
\usepackage{amssymb,amsmath}
\usepackage{ifxetex,ifluatex}
\usepackage{fixltx2e} % provides \textsubscript
% use microtype if available
\IfFileExists{microtype.sty}{\usepackage{microtype}}{}
\ifnum 0\ifxetex 1\fi\ifluatex 1\fi=0 % if pdftex
  \usepackage[utf8]{inputenc}
\else % if luatex or xelatex
  \usepackage{fontspec}
  \ifxetex
    \usepackage{xltxtra,xunicode}
  \fi
  \defaultfontfeatures{Mapping=tex-text,Scale=MatchLowercase}
  \newcommand{\euro}{€}
\fi
\ifxetex
  \usepackage[setpagesize=false, % page size defined by xetex
              unicode=false, % unicode breaks when used with xetex
              xetex]{hyperref}
\else
  \usepackage[unicode=true]{hyperref}
\fi
\hypersetup{breaklinks=true,
            bookmarks=true,
            pdfauthor={},
            pdftitle={},
            colorlinks=true,
            urlcolor=blue,
            linkcolor=magenta,
            pdfborder={0 0 0}}
\setlength{\parindent}{0pt}
\setlength{\parskip}{6pt plus 2pt minus 1pt}
\setlength{\emergencystretch}{3em}  % prevent overfull lines
\setcounter{secnumdepth}{0}
\usepackage[vmargin=1in,hmargin=1in]{geometry}

\author{}
\date{}

\begin{document}

\section{git can facilitate greater reproducibility and increased
transparency in science.}

\textbf{Karthik Ram}, Ph.D.\\Environmental Science, Policy, and
Management.\\University of California, Berkeley.\\Berkeley, CA 94720.
USA.\\\href{mailto:karthik.ram@berkeley.edu}{karthik.ram@berkeley.edu}

\subsection{Abstract}

Reproducibility is the hallmark of good science. Maintaining a high
degree of transparency in scientific reporting is essential not just for
gaining trust and credibility within the scientific community but also
for facilitating the development of new ideas. Sharing data and computer
code associated with publications is becoming increasingly common,
motivated partly in response to data deposition requirements from
journals and mandates from funders. Despite this increase in
transparency, it is still difficult to reproduce or build upon the
findings of most scientific publications without access to a more
complete workflow.

Version control systems (VCS), which have long been used to maintain
code repositories in the software industry, are now finding new
applications in science. One such open source VCS, git, provides a
lightweight yet robust framework that is ideal for managing the full
suite of research outputs such as datasets, statistical code, figures,
lab notes, and manuscripts. For individual researchers, git provides a
powerful way to track and compare versions, retrace errors, explore new
approaches in a structured manner, while maintaining a full audit trail.
For larger collaborative efforts, git and git hosting services make it
possible for everyone to work asynchronously and merge their
contributions at any time, all the while maintaining a complete
authorship trail. In this paper I provide an overview of git along with
use-cases that highlight how this tool can be leveraged to make science
more reproducible and transparent, foster new collaborations, and
support novel uses.

\section{Keywords}

reproducible research, version control, open science.

\newpage

\section{Background}

Reproducible science provides the critical standard by which published
results are judged and central findings are either validated or refuted
{[}1{]}. Reproducibility also allows others to build upon existing work
and use it to test new ideas and develop methods. Advances over the
years have resulted in the development of complex methodologies that
allow us to collect ever increasing amounts of data. While repeating
expensive studies to validate findings is often difficult, a whole host
of other reasons have contributed to the problem of reproducibility
{[}2,3{]}. One such reason has been the lack of detailed access to
underlying data and statistical code used for analysis, which can
provide opportunities for others to verify findings {[}4,5{]}. In an era
rife with costly retractions, scientists have an increasing burden to be
more transparent in order to maintain their credibility {[}6{]}. While
post-publication sharing of data and code is on the rise, driven in part
by funder mandates and journal requirements {[}7{]}, access to such
research outputs is still not very common {[}8,9{]}. By sharing detailed
and versioned copies of one's data and code researchers can not only
ensure that reviewers can make well-informed decisions, but also provide
opportunities for such artifacts to be repurposed and brought to bear on
new research questions.

Opening up access to the data and software, not just the final
publication, is one of goals of the open science movement. Such sharing
can lower barriers and serve as a powerful catalyst to accelerate
progress. In the era of limited funding, there is a need to leverage
existing data and code to the fullest extent to solve both applied and
basic problems. This requires that scientists share their research
artifacts more openly, with reasonable licenses that encourage fair use
while providing credit to original authors {[}10{]}. Besides overcoming
social challenges to these issues, existing technologies can also be
leveraged to increase reproducibility.

All scientists use version control in one form or another at various
stages of their research projects, from the data collection all the way
to manuscript preparation. This process is often informal and haphazard,
where multiple revisions of papers, code, and datasets are saved as
duplicate copies with uninformative file names (e.g. \emph{draft\_1.doc,
draft\_2.doc}). As authors receive new data and feedback from peers and
collaborators, maintaining those versions and merging changes can result
in an unmanageable proliferation of files. One solution to these
problems would be to use a formal Version Control System (VCS), which
have long been used in the software industry to manage code. A key
feature common to all types of VCS is that ability save versions of
files during development along with informative comments which are
referred to as commit messages. Every change and accompanying notes are
stored independent of the files, which obviates the need for duplicate
copies. Commits serve as checkpoints where individual files or an entire
project can be safely reverted to when necessary. Most traditional VCS
are centralized which means that they require a connection to a central
server which maintains the master copy. Users with appropriate
privileges can check out copies, make changes, and upload them back to
the server.

Among the suite of version control systems currently available,
\textbf{git} stands out in particular because it offers features that
make it ideal for managing artifacts of scientific research. The most
compelling feature of git is its decentralized and distributed nature.
Every copy of a git repository can serve either as the server (a central
point for synchronizing changes) or as a client. This ensures that there
is no single point of failure. Authors can work asynchronously without
being connected to a central server and synchronize their changes when
possible. This is particularly useful when working from remote field
sites where internet connections are often slow or non-existent. Unlike
other VCS, every copy of a git repository carries a complete history of
all changes, including authorship, that can be viewed and searched by
anyone. This feature allows new authors to build from any stage of a
versioned project. git also has a small footprint and nearly all
operations occur locally.

By using a formal VCS, researchers can not only increase their own
productivity but also make it for others to fully understand, use, and
build upon their contributions. In the rest of the paper I describe how
git can be used to manage common science outputs and move on to
describing larger use-cases and benefits of this workflow. Readers
should note that I do not aim to provide a comprehensive review of
version control systems or even git itself. My goal here is to broadly
outline some of advantages of using one such system and how it can
benefit individual researchers, collaborative efforts, and the wider
research community.

\section{Results}

\subsection{How git can track various artifacts of a research effort}

Before delving into common use-cases, I first describe how git can be
used to manage familiar research outputs such as data, code used for
statistical analyses, and documents. git can be used to manage them not
just separately but also in various combinations for different use cases
such as maintaining lab notebooks, lectures, datasets, and manuscripts.

\subsubsection{Manuscripts and notes}

Version control can operate on any file type including ones most
commonly used in academia such as Microsoft Word. However, since these
file types are binary, git cannot examine the contents and highlight
sections that have changed between revisions. In such cases, one would
have to rely solely on commit messages or scan through file contents.
The full power of git can best be leveraged when working with plain-text
files. These include data stored in non-proprietary spreadsheet formats
(e.g.~comma separated files versus \texttt{xls}), scripts from
programming languages, and manuscripts stored in plain text formats
(\texttt{LaTeX} and \texttt{markdown} versus Word documents). With such
formats, git not only tracks versions but can also highlight which
sections of a file have changed.\\In Microsoft Word documents the
\emph{track changes} feature is often used to solicit comments and
feedback. Once those comments and changes have either been accepted or
rejected, any record of their existence also disappears forever. When
changes are submitted using git, a permanent record of author
contributions remains in the version history and available in every copy
of the repository.

\subsubsection{Datasets}

Data are ideal for managing with git. These include data manually
entered via spreadsheets, recorded as part of observational studies, or
ones retrieved from sensors (see also section on \emph{Managing large
data}). With each significant change or additions, commits can record a
log those activities (e.g. ``\emph{Entered data collected between
12/10/2012 and 12/20/2012}'', or ``\emph{Updated data from temperature
loggers for December 2012}''). Over time this process avoids
proliferation of files, while the git history maintains a complete
provenance that can be reviewed at any time. When errors are discovered,
earlier versions of a file can be reverted without affecting other
assets in the project.

\subsubsection{Statistical code and figures}

When data are analyzed programmatically using software such as
\texttt{R} and \texttt{Python}, code files start out small and often
become more complex over time. Somewhere along the process, inadvertent
errors such as misplaced subscripts and incorrectly applied functions
can lead to serious errors down the line. When such errors are
discovered well into a project, comparing versions of statistical
scripts can provide a way to quickly trace the source of the problem and
recover from them.

Similarly, figures that are published in a paper often undergo multiple
revisions before resulting in a final version that gets published.
Without version control, one would have to deal with multiple copies and
use imperfect information such as file creation dates to determine the
sequence in which they were generated. Without additional information,
figuring out why certain versions were created (e.g.~in response to
comments from coauthors) also becomes more difficult. When figures are
managed with git, the commit messages (e.g. ``\emph{Updated figure in
response to Ethan's comments regarding use of normalized data.}'')
provide an unambiguous way to track various versions.

\subsubsection{Complete manuscripts}

When all of the above artifacts are used in a single effort, such as
writing a manuscript, git can collectively manage versions in a powerful
way for both individual authors and groups of collaborators. This
process avoids rapid multiplication of unmanageable files with
uninformative names (e.g. \emph{final\_1.doc, final\_2.doc,
final\_final.doc, final\_KR\_1.doc} etc.) as
\href{http://www.phdcomics.com/comics/archive.php?comicid=1531}{illustrated}
by the popular cartoon strip PhD Comics.

\section{Use cases for git in science}

\subsection{1. Lab notebook}

Day to day decisions made over the course of a study are often logged
for review and reference in lab notebooks. Such notebooks contain
important information useful to both future readers attempting to
replicating a study, or for thorough reviewers seeking additional
clarification. However, lab notebooks are rarely shared along with
publications or made public although there are some exceptions {[}11{]}.
Git commit logs can serve as a proxies for lab notebooks if clear yet
concise messages are recorded over the course of a project. One of the
fundamental features of git that make it so useful to science is that
every copy of a repository carries a complete history of changes
available for anyone to review. These logs can be be easily searched to
retrieve versions of artifacts like data and code. Third party tools can
also be leveraged to mine git histories from one or more projects for
other types of analyses.

\subsection{2. Facilitating Collaboration}

In collaborative efforts, authors contribute to one or more stages of
the manuscript preparation such as collecting data, analyzing them,
and/or writing up the results. Such information is extremely useful for
both readers and reviewers when assessing relative author contributions
to a body of work. With high profile journals now discouraging the
practice of honorary authorship {[}12{]}, git commit logs can provide a
highly granular way to track and assess individual author contributions
to a project.

When projects are tracked using git, every single action (such as
additions, deletions, and changes) is attributed to an author. Multiple
authors can choose to work on a single branch of a repository (the
`\emph{master}' branch), or in separate branches and work
asynchronously. In other words, authors do not have to wait on coauthors
before contributing. As each author adds their contribution, they can
sync those to the master branch and update their copies at any time.
Over time, all of the decisions that go into the production of a
manuscript from entering data and checking for errors, to choosing
appropriate statistical models and creating figures, can be traced back
to specific authors.

With the help of a remote git hosting services, maintaining various
copies in sync with each other becomes effortless. While most changes
are merged automatically, conflicts will need to be resolved manually
which would also be the case with most other workflows (e.g.~using
Microsoft Word with track changes). By syncing changes back and forth
with a remote repository, every author can update their local copies as
well as push their changes to the remote version at any time, all the
while maintaining a complete audit trail. Mistakes or unnecessary
changes can easily undone by reverting either the entire repository or
individual files to earlier commits. Since commits are attributed to
specific authors, error or clarifications can also be appropriately
directed. Perhaps most importantly this workflow ensures that revisions
do not have to be emailed back and forth. While cloud storage providers
like Dropbox alleviate some of these annoyances and also provide
versioning, the process is not controlled making it hard to discern what
and how many changes have occurred between two time intervals.

In a recent paper led by Philippe Desjardins-Proulx
\href{https://github.com/PhDP/article\_preprint/network}{https://github.com/PhDP/article\_preprint/network}
all of the authors successfully collaborated using only git and GitHub
(\href{{[}@Vink2012b{]}}{https://github.com/}). In this particular git
workflow, each of us cloned a copy of the main repository and
contributed our changes back to the lead author. Figures \texttt{2} and
\texttt{3} show the list of collaborators and a network diagram of how
and when changes were contributed back the master branch.

\subsection{3. Backup and failsafe against data loss}

Collecting new data and developing methods for analysis are often
expensive endeavors requiring significant amounts of grant funding.
Therefore protecting such valuable products from loss or theft is
paramount. A recent study found that a vast majority of data and code
are stored on lab computers or web servers both of which are prone to
failure and often become inaccessible after a certain length of time.
One survey found that only 72\% of studies of 1000 surveyed still had
data that were accessible {[}13,14{]}. Hosting data and code publicly
not only ensures protection against loss but also increases visibility
for research efforts and provides opportunities for collaboration and
early review {[}15{]}.

While git provides a powerful features that can leveraged by individual
scientists, git hosting services open up a whole new set of
possibilities. Any local git repository can be linked to one or more
\textbf{git remotes}, which are copies hosted on a remote cloud severs.
Git remotes serve as hubs for collaboration where authors with write
privileges can contribute anytime while others can download up-to-date
versions or submit revisions with author approval. There are currently
several git hosting services such as SourceForge, Google Code, GitHub,
and BitBucket that provide free git hosting. Among them, Github has
surpassed other popular provides like Google Code and SourceForge and
hosts over 2 million public repositories at the time of this writing
{[}16,17{]}. While these services are usually free for publicly open
projects, some research efforts, especially those containing embargoed
or sensitive data will need to be kept private. There are multiple ways
to deal with such situations. For example, certain files can be excluded
from git's history, others maintained as private sub-modules, or entire
repositories can be made private and opened to the public at a future
time. Some git hosts like BitBucket offer unlimited public and private
accounts for academic use.

Managing a research project with git provides several safe guards
against short-term loss. Frequent commits synced to remote repositories
ensure that multiple versioned copies are accessible from anywhere. In
projects involving multiple collaborators, the presence of additional
copies makes even more difficult to lose work. While git hosting
services protect against short-term data loss, they are not a solution
for more permanent archiving since none of them offer any such
guarantees. For long-term archiving, researchers should submit their
git-managed projects to academic repositories that are members of
CLOCKSS (\href{http://www.clockss.org/}{http://www.clockss.org/}).
Output stored on such repositories (e.g.~figshare) are archived over a
network of redundant nodes and ensure indefinite availability across
geographic and geopolitical regions.

\subsection{4. Freedom to explore new ideas and methods}

git tracks development of projects along timelines referred to as
\textbf{\emph{branches}}. By default, there is always a master branch
(line with blue dots in figure \texttt{1}). For most authors, working
with this single branch is sufficient. However, git provides a powerful
branching mechanism that makes it easy for exploring alternate ideas in
a structured and documented way without disrupting the central flow of a
project. For example, one might want to try an improved simulation
algorithm, a novel statistical method, or plot figures in a more
compelling way. If these changes don't work out, one could revert
changes back to an earlier commit when working on a single master
branch. Frequent reverts on a master branch can be disruptive,
especially when projects involve multiple collaborators. Branching
provides a risk-free way to test new algorithms, explore better data
visualization techniques, or develop new analytical models. When
branches yield desired outcomes, they can easily be merged into the
master copy while unsuccessful efforts can be deleted or left as-is to
serve as a historical record (illustrated in figure \texttt{1}).

Branches can prove extremely useful when responding to reviewer
questions about the rationale for choosing one method over another since
the git history contains a record of failed, unsuitable, or abandoned
attempts. This is particularly helpful given that the time between
submission and response can be fairly long. Additionally, future users
can mine git histories to avoid repeating approaches that were never
fruitful in earlier studies.

\subsection{5. Mechanism to solicit feedback and reviews}

While it is possible to leverage most of core functionality in git at
the local level, git hosting services offer additional services such as
issue trackers, collaboration graphs, and wikis. These can easily be
used to assign tasks, manage milestones, and maintain lab protocols.
Issue trackers can be repurposed as a mechanism for soliciting both
feedback and review, especially since the comments can easily be linked
to particular lines of code or blocks of text. Early comments and
reviews for this article were also solicited via GitHub Issues
\href{https://github.com/karthikram/smb\_git/issues/}{https://github.com/karthikram/smb\_git/issues/}

\subsection{6. Increase transparency and verifiability}

Methods sections in papers are often succinct to adhere to strict word
limits imposed by journal guidelines. This practice is especially common
when describing well-known methods where authors assume a certain degree
of familiarity among informed readers. One unfortunate consequence of
this practice is that any modifications to the standard protocol
(typically noted in internal lab notebooks) implemented in a study may
not available to the reviewers and readers. However, seemingly small
decisions, such as choosing an appropriate distribution to use in a
statistical method, can have a disproportionately strong influence on
the central finding of a paper. Without access to a detailed history, a
reviewer competent in statistical methods has to trust that authors
carefully met necessary assumptions, or engage in a long back and forth
discussion thereby delaying the review process. Sharing a git repository
can alleviate these kinds of ambiguities and allow authors to point out
commits where certain key decisions were made before choosing certain
approaches. Journals could facilitate this process by allowing authors
to submit links to their git repository alongside manuscripts and
sharing them with reviewers.

\subsection{7. Managing large data}

git is extremely efficient with managing small data files such as ones
routinely collected in experimental and observational studies. However,
when the data are particularly large such as those in bioinformatics
studies (in the order of tens of megabytes to gigabytes), managing them
with git can degrade efficiency and slow down the performance of git
operations. With large data files, the best practice would be to exclude
them from the repository and only track changes in metadata. This
protocol is especially ideal when large datasets do not change often
over the course of a study. In situations where the data are large
\emph{and} undergo frequent updates, one could leverage third-party
tools such as git-annex
\href{http://git-annex.branchable.com/}{http://git-annex.branchable.com/}
and still seamlessly use git to manage a project.

\subsection{8. Lowering barriers to reuse}

A common barrier that prevents someone from reproducing or building upon
an existing method is lack of sufficient details about a method. Even in
cases where methods are adequately described, the use of expensive
proprietary software with restrictive licenses makes it difficult to use
{[}18{]}. Sharing code with licenses that encourage fair use with
appropriate attribution removes such artificial barriers and encourages
readers to modify methods to suit their research needs, improve upon
them, or find new applications {[}10{]}. With open source software,
analysis pipelines can be easily \emph{forked} or branched from public
git repositories and modified to answer other questions. Although this
process of depositing code somewhere public with appropriate licenses
involves additional work for the authors, the overall benefits outweigh
the costs. Making all research products publicly available not only
increases citation rates {[}19--21{]} but can also increase
opportunities for collaboration by increasing overall visibility. For
example, Niedermeyer \& Strohalm {[}22{]} describe their struggle with
finding appropriate software for comprehensive mass spectrum annotation,
and eventually found an open source software which they where able to
extend. In particular, the authors cite availability of complete source
code along with an open license as the motivation for their choice.
Examples of such collaboration and extensions are likely to become more
common with increased availability of fully versioned projects with
permissive licenses.

A similar argument can be made for data as well. Even publications that
deposit data in persistent repositories rarely share the original raw
data. The versions submitted to persistent repositories are often
\emph{cleaned} and finalized versions of datasets. In cases where no
datasets are deposited, the only data accessible are likely mean values
reported in the main text or appendix of a paper. Raw data can be
leveraged to answer questions not originally intended by the authors.
For example, research areas that address questions about uncertainly
often require messy raw data to test competing methods. Thus, versioned
data provide opportunities to retrieve copies before they have been
modified for use in different contexts and have lost some of their
utility.

\section{Conclusions}

Wider use of git has the potential to revolutionize scholarly
communication and increase opportunities for reuse, novel synthesis, and
new collaborative efforts. With disciplined use of git, individual
scientists and labs can ensure that the entire timeline of events that
occur over the development of a research project are securely logged in
a system that provides security against data loss and encourages
risk-free exploration of new ideas and approaches. In an era with
shrinking research budgets, scientists are under increasing pressure to
produce more with less. If more granular sharing via git reduces time
spent developing new software, or repeating expensive data collection
efforts, then everyone stands to benefit. Scientists should note that
these efforts don't have to viewed as entirely altruistic. In a recent
mandate the National Science Foundation {[}23{]} has expanded its merit
guidelines to include a range of academic products such as software and
data, in addition to peer-reviewed publications. With the rise in use of
altmetric tools that track and credit such efforts, then everyone can
benefit {[}24{]}.

Although I have laid out various arguments for why more scientists
should be using git, one should be careful not to view git as a one stop
solution to all the problems facing reproducibility in science. Although
the basic features of git can be readily used without any knowledge of
command line tools, leveraging the full power of git, especially when
working on complex projects where one might encounter unwieldy merge
conflicts, comes at a significant learning cost. There are also
comparable alternatives to git (e.g.~Mercurial) which offer less
granularity but are more user-friendly. While time invested in becoming
proficient in git would be valuable in the long-term, most scientists do
not have the luxury of learning software skills that do not address more
immediate problems. Despite the fact that scientists spent considerable
time using and creating their own software to address domain specific
needs, good programming practices are rarely taught {[}25{]}. Therefore
wider adoption of useful tools like git will require greater software
development literacy among scientists. On a more optimistic note, such
literacy is slowly becoming common in the new generation of academics,
driven in part by efforts such as Software Carpentry
\href{http://software-carpentry.org/}{http://software-carpentry.org/}
and newer courses taught in graduate curricula (e.g.
\href{http://www.programmingforbiologists.org/}{Programming for
biologists} taught at Utah State University).

\section{List of Abbreviations}

VCS: Version Control System; NSF: National Science Foundation; CSV:
Comma Separated Values.

\subsection{Acknowledgements}

Comments from Carl Boettiger, Yoav Ram, David Jones, and Scott
Chamberlain on earlier drafts greatly improved the final version of this
article. This manuscript is available both as a git repository (with a
full history of changes)
\href{https://github.com/karthikram/smb\_git.git}{https://github.com/karthikram/smb\_git.git}
and also as a permanent archived copy on figshare (http://figshare.com/)
(I'll add a link to figshare URL once a final version of the paper is
accepted). I also thank the rOpenSci project
(\href{http://ropensci.org}{http://ropensci.org}) for helping me gain a
greater appreciation for git as a tool for advancing science.

\section{Author contributions}

KR conceived and wrote the manuscript. The author has read and approved
the manuscript.

\section{Competing interests}

I declare that I have no competing interests.

\section{Funding support}

The author did not receive any specific funding for this work.

\subsection{Literature Cited}

1. Vink CJ, Paquin P, Cruickshank RH (2012) Taxonomy and Irreproducible
Biological Science. BioScience 62: 451--452. Available:
\href{http://www.bioone.org/doi/abs/10.1525/bio.2012.62.5.3}{http://www.bioone.org/doi/abs/10.1525/bio.2012.62.5.3}.

2. Peng RD (2011) Reproducible Research in Computational Science.
Science 334: 1226--1227. Available:
\href{http://www.sciencemag.org/cgi/doi/10.1126/science.1213847}{http://www.sciencemag.org/cgi/doi/10.1126/science.1213847}.

3. Begley CG, Ellis LM (2012) Drug development: Raise standards for
preclinical cancer research. Nature 483: 531--3. Available:
\href{http://dx.doi.org/10.1038/483531a}{http://dx.doi.org/10.1038/483531a}.

4. Schwab M, Karrenbach M, Claerbout J (2000) Making Scientific
Computations Reproducible. Computing in Science Engineering 2: 61--67.
Available:
\href{http://ieeexplore.ieee.org/lpdocs/epic03/wrapper.htm?arnumber=881708}{http://ieeexplore.ieee.org/lpdocs/epic03/wrapper.htm?arnumber=881708}.

5. Ince DC, Hatton L, Graham-Cumming J (2012) The case for open computer
programs. Nature 482: 485--8. Available:
\href{http://dx.doi.org/10.1038/nature10836}{http://dx.doi.org/10.1038/nature10836}.

6. Van Noorden R (2011) The trouble with retractions. Nature: 6--8.

7. Whitlock MC, McPeek MA, Rausher MD, Rieseberg L, Moore AJ (2010) Data
archiving. The American naturalist 175: 145--6. Available:
\href{http://www.jstor.org/stable/10.1086/650340}{http://www.jstor.org/stable/10.1086/650340}.

8. Vines TH, Andrew RL, Bock DG, Franklin MT, Gilbert KJ, et al. (2013)
Mandated data archiving greatly improves access to research data. FASEB
journal official publication of the Federation of American Societies for
Experimental Biology. doi:10.1096/fj.12-218164

9. Wolkovich EM, Regetz J, O'Connor MI (2012) Advances in global change
research require open science by individual researchers. Global Change
Biology 18: 2102--2110. Available:
\href{http://apps.webofknowledge.com/full\textbackslash{}\_record.do?product=UA\textbackslash{}\&search\textbackslash{}\_mode=GeneralSearch\textbackslash{}\&qid=1\textbackslash{}\&SID=1CfaPnJ9gbl5bo171Jc\textbackslash{}\&page=1\textbackslash{}\&doc=4}{http://apps.webofknowledge.com/full\textbackslash{}\_record.do?product=UA\textbackslash{}\&search\textbackslash{}\_mode=GeneralSearch\textbackslash{}\&qid=1\textbackslash{}\&SID=1CfaPnJ9gbl5bo171Jc\textbackslash{}\&page=1\textbackslash{}\&doc=4}.

10. Neylon C (2013) Open access must enable open use. Nature 492: 8--9.

11. Wald C (2010) Issues \& Perspectives Scientists Embrace Openness.
Available:
\href{http://sciencecareers.sciencemag.org/career\textbackslash{}\_magazine/previous\textbackslash{}\_issues/articles/2010\textbackslash{}\_04\textbackslash{}\_09/caredit.a1000036}{http://sciencecareers.sciencemag.org/career\textbackslash{}\_magazine/previous\textbackslash{}\_issues/articles/2010\textbackslash{}\_04\textbackslash{}\_09/caredit.a1000036}.
Accessed 16 Jan 2013.

12. Greenland P, Fontanarosa PB (2012) Ending honorary authorship.
Science (New York, N.Y.) 337: 1019. Available:
\href{http://www.sciencemag.org/content/337/6098/1019.short}{http://www.sciencemag.org/content/337/6098/1019.short}.

13. Schultheiss SJ, Münch M-C, Andreeva GD, Rätsch G (2011) Persistence
and availability of Web services in computational biology. PloS one 6:
24914. Available:
\href{http://dx.plos.org/10.1371/journal.pone.0024914}{http://dx.plos.org/10.1371/journal.pone.0024914}.

14. Wren JD (2004) 404 not found: the stability and persistence of URLs
published in MEDLINE. Bioinformatics (Oxford, England) 20: 668--72.
Available:
\href{http://bioinformatics.oxfordjournals.org/content/20/5/668.abstract}{http://bioinformatics.oxfordjournals.org/content/20/5/668.abstract}.

15. Prlić A, Procter JB (2012) Ten Simple Rules for the Open Development
of Scientific Software. PLoS Computational Biology 8: 1002802.
Available:
\href{http://dx.plos.org/10.1371/journal.pcbi.1002802}{http://dx.plos.org/10.1371/journal.pcbi.1002802}.

16. Pearson DP (2013) GitHub sees 3 millionth member account. Available:
\href{http://www.gamesindustry.biz/articles/2013-01-17-github-sees-3-millionth-member-account}{http://www.gamesindustry.biz/articles/2013-01-17-github-sees-3-millionth-member-account}.
Accessed 18 Jan 2013.

17. Finley K (2011) Github Has Surpassed Sourceforge and Google Code in
Popularity. Available:
\href{http://readwrite.com/2011/06/02/github-has-passed-sourceforge}{http://readwrite.com/2011/06/02/github-has-passed-sourceforge}.
Accessed 15 Jan 2013.

18. Morin A, Urban J, Sliz P (2012) A quick guide to software licensing
for the scientist-programmer. PLoS computational biology 8: 1002598.
Available:
\href{http://dx.plos.org/10.1371/journal.pcbi.1002598}{http://dx.plos.org/10.1371/journal.pcbi.1002598}.

19. Piwowar HA, Day RS, Fridsma B (2007) Sharing Detailed Research Data
Is Associated with Increased Citation Rate. PLOS One.

20. Piwowar H a (2011) Who shares? Who doesn't? Factors associated with
openly archiving raw research data. PloS one 6: 18657. Available:
\href{http://www.pubmedcentral.nih.gov/articlerender.fcgi?artid=3135593\textbackslash{}\&tool=pmcentrez\textbackslash{}\&rendertype=abstract}{http://www.pubmedcentral.nih.gov/articlerender.fcgi?artid=3135593\textbackslash{}\&tool=pmcentrez\textbackslash{}\&rendertype=abstract}.

21. Alsheikh-Ali A a, Qureshi W, Al-Mallah MH, Ioannidis JP a (2011)
Public availability of published research data in high-impact journals.
PloS one 6: 24357. Available:
\href{http://www.pubmedcentral.nih.gov/articlerender.fcgi?artid=3168487\textbackslash{}\&tool=pmcentrez\textbackslash{}\&rendertype=abstract}{http://www.pubmedcentral.nih.gov/articlerender.fcgi?artid=3168487\textbackslash{}\&tool=pmcentrez\textbackslash{}\&rendertype=abstract}.

22. Niedermeyer THJ, Strohalm M (2012) mMass as a software tool for the
annotation of cyclic peptide tandem mass spectra. PloS one 7: 44913.
Available:
\href{http://www.pubmedcentral.nih.gov/articlerender.fcgi?artid=3441486\textbackslash{}\&tool=pmcentrez\textbackslash{}\&rendertype=abstract}{http://www.pubmedcentral.nih.gov/articlerender.fcgi?artid=3441486\textbackslash{}\&tool=pmcentrez\textbackslash{}\&rendertype=abstract}.

23. US NSF - Dear Colleague Letter - Issuance of a new NSF Proposal \&
Award Policies and Procedures Guide (NSF13004) (2012) US NSF - Dear
Colleague Letter - Issuance of a new NSF Proposal \& Award Policies and
Procedures Guide (NSF13004). Available:
\href{http://www.nsf.gov/pubs/2013/nsf13004/nsf13004.jsp?WT.mc\textbackslash{}\_id=USNSF\textbackslash{}\_109}{http://www.nsf.gov/pubs/2013/nsf13004/nsf13004.jsp?WT.mc\textbackslash{}\_id=USNSF\textbackslash{}\_109}.
Accessed 11 Nov 2012.

24. Piwowar H (2013) Altmetrics: Value all research products. Nature
493: 159--159. Available:
\href{http://www.nature.com/doifinder/10.1038/493159a}{http://www.nature.com/doifinder/10.1038/493159a}.

25. Wilson G, Aruliah DA, Brown CT, Hong NPC, Davis M, et al. (2012)
Best Practices for Scientific Computing. Arxiv: 6.

\end{document}
